\section{实验要求}
\subsection{数据预处理:分词}
从原始的小说文本中,抽取与人物互动相关的数据,屏蔽掉与人物关系无关的文本内容,为后面基于人物共现的分析做准备。\\
\textbf{输入}:1. 金庸全本武侠小说文集(未分词);2. 金庸武侠小说人名列表。\\
\textbf{输出}:分词后,仅保留人名的武侠小说全集。
\subsection{特征提取:单词同现}
在人物同现分析中,如果两个人在原文的同一段落中出现,则认为两个人发生了一次同现关系。需要对人物之间的同现关系次数进行统计,同现关系次数越多,则说明两人关系越密切。\\
\textbf{输入}:分词后仅保留人名的武侠小说全集。\\
\textbf{输出}:在金庸的所有武侠小说中,人物之间的同现次数。
\subsection{归一化}
获得人物之间的同现次数后,根据同现关系生成人物之间的关系图。在关系图中,人物是顶点,人物之间的互动关系是边。边的权值是两个人之间的共现概率。\\
\textbf{输入}:人物共现次数矩阵。\\
\textbf{输出}:权重归一化后的人物关系图。
\subsection{PageRank}
获得权重归一化化后的人物关系图后,可以进行数据分析任务。PageRan是典型的数据分析任务,通过PageRank可以确定金庸武侠江湖中的“主角”是那些。\\
\textbf{输入}:权重归一化后的人物关系图。
\\\textbf{输出}:人物的PageRank值。
\subsection{标签传播}
标签传播(Label Propagation)是一种半监督的图分析算法,他能为图上的顶点打上标签,进行图顶点的聚类分析。通过在人物关系图上进行标签传播,可以将属于同一本书的人物聚为一类。\\
\textbf{输入}:权重归一化后的人物关系图。\\
\textbf{输出}:人物的标签信息。
\subsection{输出整理}
对于PageRank的结果,可以使用MapReduce任务按照PageRank值进行全局排序。\\
对于标签传播的结果,可以使用MapReduce任务将属于同一标签的任务输出到一起,便于结果查看。
\section{小组分工}
\begin{table}[!htbp]
	\centering
	\caption{小组分工}
	\begin{tabular} {|p{50pt}|p{80pt}|p{250pt}|}
		\hline
		姓名& 学号 &任务\\
		\hline
		朱庭纬&161220186&数据预处理,特征提取,归一化,报告撰写\\
		\hline
		崔子寒&161220026&PageRank,结果整理,报告撰写\\
		\hline
		吴昌容&161220134&标签传播任务,结果整理,报告撰写\\
		\hline
	\end{tabular}
\end{table}
\section{实验环境}
\begin{table}[!htbp]
	\centering
	\caption{实验环境}
	\begin{tabular} {|p{80pt}|p{160pt}|}
		\hline
		操作系统&Ubuntu 18.04 LTS\\
		\hline
		hadoop版本& 2.7.1\\
		\hline
		jdk版本& 1.7.0\\
		\hline
		构建工具&Maven\\
		\hline
	\end{tabular}
\end{table}
\section{编译、打包、运行}
\subsection{编译打包}
由于使用Maven作为构建工具,所以可以很方便的添加依赖,我们在pom.xml文件中添加了表3所示的依赖:
\begin{table}[!htbp]
	\centering
	\caption{项目依赖}
	\begin{tabular} {|p{90pt}|p{250pt}|p{50pt}|}
		\hline
		groupId& artifactId &version\\
		\hline
		org.apache.hadoop&hadoop-mapreduce-client-core&2.6.0\\
		\hline
		org.apache.hadoop&hadoop-test&1.2.1\\
		\hline
		org.apache.hadoop&hadoop-common&2.7.1\\
		\hline
		org.ansj&ansj\_seg&5.1.6\\
		\hline
	\end{tabular}
\end{table}
\\
其中,groupId为org.apache.hadoop的三个依赖是hadoop MapReduce API的有关依赖。ansj\_seg是分词库的依赖,这里使用的是官网上最新的5.1.6版本。\\
为了在打包时将第三方的依赖和代码一起打包,使用Maven插件org.apache.maven.plugins: maven-shade-plugin,通过设置目标为shade,可以将依赖连同代码一起打包。打包的命令为:\textbf{mvn package -Dhttps.protocols=TLSv1.2}。 由于我们使用的jdk版本是1.7,而1.7默认的https协议是TLSv1.1,现在maven仓库已经不再支持这个版本的协议,所以需要加上参数,否则在下载依赖时会出错。
\subsection{运行方式}
任务代码由三名成员合作编写,每人负责一个模块,然后将模块组合起来。运行时既可以通过运行主类Main,依次运行所有任务,也可以分别运行某一个子任务。具体的运行方式如表4所示(假设jar包名称为jianghu.jar):
\begin{table}[!htbp]
	\centering
	\caption{运行方式}
	\begin{tabular} {|p{180pt}|p{300pt}|}
		\hline
		指令&解释\\
		\hline
		hadoop jar jianghu.jar cn.nju.st13.Main arg0 arg1 arg2&arg0:小说文件夹路径;arg1:人名文件路径;arg2:输出结果路径\\
		\hline
		hadoop jar  jianghu.jar  cn.nju.st13.preprocess.Preprocess arg0 arg1 arg2 arg3&arg0:小说文件夹路径;arg1:人名文件路径;arg2:人物关系图输出路径;arg3:可选参数"-1","-2","-3",代表三种不同的方法\\
		\hline
		hadoop jar jianghu.jar cn.nju.st13.pagerank.PageRankDriver arg0 arg1 可选参数&arg0:人物关系图路径;arg1:排序处理后的PageRank结果路径;可选参数:-max\_iter=n(n为最大迭代次数,默认15),-retain\_process=false/true(是否保留中间结果,默认false)\\
		\hline
		hadoop jar jianghu.jar cn.nju.st13.labelprop.LabelProp arg0 arg1&arg0:人物关系图路径;arg1:整理后的标签传播算法结果路径。\\
		\hline
	\end{tabular}
\end{table}
