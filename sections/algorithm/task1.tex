\subsection{任务1}
首先我们需要从原始文本中把文章中所有的人名抽取出来,屏蔽掉其他与人物关系分析任务无关的内容。要完成此任务有两个步骤:将段落分割成词语,再从所有词语中过滤出金庸小说中的人物名字。\\
\indent 问题的关键在于如何正确分词,使得目标人名能够被正确提取。我们使用的分词工具是Ansj\_seg,在分词前向词典中添加所有自定义词语即可。分词后判断每个词语的词性是否为“nr”(即人名)以及是否在给定的人名列表中即可提取出所有人物名称。\\
\indent 由于每行的处理方式均相同,仍可采用mapreduce进行并行处理。在mapper端代码中,在setup函数中读取人名列表,将列表中每个人名添加到词典中;在map函数中,对于每一行,将分词结果存入Result类型变量中,然后根据词性和是否在人名列表中判断每个词是否为目标人名。最后将所有目标人名用空格连接并emit。无需Reducer端。\\
\indent 人名列表是需要全局共享的变量。由于数据量不大,在本任务中,使用 Job 的 Configuration 在 Driver 和 Mapper 端进行数据的传递。在 Driver 端的 main 函数中,从本地的文件中读入人名列表,然后将人名列表转化为一个长字符串。通过 Configuration 的 set 方法,将人名列表信息存入 Configuration 中。再在Mapper端中的setup函数中读取即可。\\
Mapper端伪代码如下所示:
\begin{lstlisting}[language=Java]
WordSegMapper{
	List<String> nameList;
	setup(){
		nameListStr = context.getConfiguration().get("NameList");
		nameList = nameListStr.split(" ");
		for name in nameList do:
			DicLibrary.insert(name);
	}
	map(input){
		terms = parse(input);
		result = empty;
		for term in terms do:
			if term.getNatureStr == "nr" and term is in the nameList:
				add term to result;
		key = join all the word in result;
		emit(key, NullWritable);
	}
}
\end{lstlisting}