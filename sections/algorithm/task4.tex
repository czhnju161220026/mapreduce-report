\newpage

\subsection{PageRank}
\subsubsection{输入输出格式}
PageRank任务的输入是人物关系图,输出是按照PageRank值从大到小排序的<人名,PageRank值>键值对。算法的输入为归一化后的人物关系图,输入文件具有如下的格式:
\begin{center}
	\fbox{\shortstack[l]{丁同\ 	李三,0.250000|李文秀,0.500000|老头子,0.125000|霍元龙,0.125000 \\
			乔福\ 	汉子,0.062500|小凤,0.312500|张无忌,0.500000|朱九真,0.125000 \\
			九死生\ 	百草仙,0.200000|韩无垢,0.200000|黄蓉,0.200000|张一氓,0.200000|人厨子,0.200000}}
\end{center}
其中每一行的键是人名,后面是该人物所有的邻居和对应的出边权值。输出文件具有如下格式:
\begin{center}
	\fbox{\shortstack[l]{韦小宝	0.027058917763952448\\
			令狐冲	0.01603786999665175\\
			张无忌	0.015589415048158133
		}}
\end{center}
\subsubsection{任务流程}
算法的大致流程描述如下:
\begin{enumerate}
	\item 初始化图结构,为每个人物分配初始的PageRank值。假设有$n$个人物,那么每个人物的PageRank值为$\frac{1}{n}$。
	\item 迭代计算每个人物的PageRank值。为了防止出现rank leak和rank sink问题,使用随机游走模型。但是和一般的随机游走模型不同的是,我们按照出边的权值分配传递PageRank值,而不是平均分配。用$W_{ij}$表示边$e_{ij}$的权值,$M(p_i)$表示指向$p_i$的节点的集合,那么PageRank值的更新公式如下:
	\[
		PR(p_i) = \frac{1-d}{N} + d \sum_{p_j \in M(p_i)} W_{ji} \times PR(p_j)
	\]
	其中,$d$为阻尼系数,常用的值为0.85。
	\item 结果整理:经过若干轮迭代,最终得到了每个人物和对应的PR值,但是还要对结果进行整理。通过一个MapReduce任务,从最后一轮迭代的结果中读取人物和对应的PR值,使用PR值作为键,并重载它的比较函数,最终得到按照PR值从大到小排序的输出
\end{enumerate}
从PageRank算法的步骤可以看出,大致需要3个MapReduce任务(实际操作中,统计人物个数$n$也用到了一个MapReduce任务,但是由于比较简单,所以不详细叙述该步骤)。分别为PageInitializer,PageIter,PageViewer。这三个MapReduce任务对应上述的三个步骤。最终我们用一个Driver将三个任务包装在一起运行,接下来依次介绍这三个任务的具体操作。

\subsubsection{PageInitializer}
这一步比较简单,由于特征归一化已经为建立起了人物关系图,所以只需要初始化每个人物结点的PR值为$\frac{1}{n}$即可。初始化在Map阶段进行,Reduce阶段不需要进行任何处理,直接输出结果即可。Map阶段的伪代码如下:
\begin{algorithm}[H]
	\caption{PageInitializerMap($name,neighbors,n$)}
	\begin{algorithmic}[1]
		\REQUIRE 人物姓名$name$,邻居表$neighbors$,人物个数$n$
		\ENSURE 初始化了$PR$的结点信息
		\STATE $Emit(name, <\frac{1}{n}, neighbors>)$
	\end{algorithmic}
\end{algorithm}
Reduce阶段什么事情都不做,只要直接输出Map的结果即可。

\subsubsection{PageIter}
这一步是Page任务中最难的一步。在Map阶段,需要按照边的权值,将$PR$值分发给邻居结点,同时也要传递图结构,以便在输出中保留图结构,供下次迭代时使用。在Reduce阶段,需要按照随机游走模型,计算新的$PR$值,连同图结构一起输出。算法的伪代码如下:
\begin{algorithm}[H]
	\caption{PageIterMap($name,<PR,neighbors>$)}
	\begin{algorithmic}[1]
		\REQUIRE 人物姓名$name$,PR值,邻居表
		\ENSURE 图结构和分发的$PR$值
		\FOR{$<neighbor,weight>$ in $neighbors$ }
		\STATE $Emit(neighbor, PR\times weight)$
		\ENDFOR
		\STATE $Emit(name, neighbors)$
	\end{algorithmic}
\end{algorithm}

\begin{algorithm}[H]
	\caption{PageIterReduce($name,[p_1,p_2...p_m],n$)}
	\begin{algorithmic}[1]
		\REQUIRE 人物姓名$name$,$Map$的输出
		\ENSURE 迭代后的结果
		\STATE $PR$:=0
		\STATE $sum$:=0
		\STATE $neighbors$:=$\emptyset$
		\FOR{$i:=1$ to $m$}
		\IF{$p_i$ is value}
		\STATE $sum := sum + p_i$
		\ELSE
		\STATE $neighbors:=p_i$
		\ENDIF
		\ENDFOR
		\STATE $PR:= \frac{1-d}{n} + d \times sum$
		\STATE $Emit(name, <PR, neighbors>)$
	\end{algorithmic}
\end{algorithm}
解释:在Map阶段,对于当前人物的每一个邻居,进行一次$Emit$,将当前人物的$PR$值按照边权值进行分发。最后再进行一次$Emit$,传递图结构。在Reduce阶段,重建图结构,计算$PR$值。可以看出,Reduce阶段的输出格式和Map阶段的输入格式完全相同,这确保了算法是可以迭代进行的。

\subsubsection{PageViewer}
这一阶段可以借助MapReduce框架中的shuffle阶段来完成。只需要重载$PR$值类型的比较器,然后将$PR$值作为键,人名作为值,使用MapReduce任务直接输出即可。

\subsubsection{实现细节}
PageInitializer任务是比较好实现的,不涉及复杂的处理。比较关键的部分是PageIter。这个子任务的难点在于如何在MapReduce中同时传递PR值和图结构。因为hadoop框架中,规定了值只能是同一种类型,所以我们自定义了一个数据结构PageRankElement,它既可以是值,也可以是图结构,它的具体实现如下:
\begin{lstlisting}[language=Java]
public class PageRankElement implements Writable {
	//如果isNode是True, 那么传递的是图结构,content是所有的邻居
	//如果isNode是False,那么传递的是pagerank值
	private boolean isNode;
	private String content;
	private double passValue;

	@Override
	public void write(DataOutput dataOutput) throws IOException {
		dataOutput.writeBoolean(isNode);
		dataOutput.writeUTF(content);
		dataOutput.writeDouble(passValue);
	}

	@Override
	public void readFields(DataInput dataInput) throws IOException {
		isNode = dataInput.readBoolean();
		content = dataInput.readUTF();
		passValue = dataInput.readDouble();
	}
	//constructor,getter and setter
	//...
}
\end{lstlisting}
在PageRankElement类中,我们覆写wirte和readFields方法,以确保我们的数据结构可以被序列化和反序列化,从而在集群中传播。通过一个boolean型变量,可以判断传递的是值还是结点。\\
使用自定义数据结构的好处是,提高了代码的可读性:虽然可以通过在字符串中添加标志位等方法来实现相同的功能,但是那样写出来的代码可读性不好,使用自定义的数据结构可以将判断,处理的步骤包装成成员函数。通过getter和setter来读写内容,使代码简单易懂。以下是PageIter的部分代码:\\
\newpage
\begin{lstlisting}[language=Java]
//Mapper类
public static class PageRankIterMapper extends Mapper<Object, Text, Text, PageRankElement> {
	@Override
	public void map(Object key, Text value, Context context) throws IOException, InterruptedException {
		...
		//传递图结构
		context.write(new Text(nodeName), new PageRankElement(neighbours.toString()));
		//分发PR值
		for (int i = 0; i < length - 1; i++) {
			String[] nameAndWeight = tokens[i].split("[,]");
			String neighbourName = nameAndWeight[0];
			double weight = Double.parseDouble(nameAndWeight[1]);
			context.write(new Text(neighbourName), new PageRankElement(weight * prValue));
	}
}

//Reducer类
public static class PageRankIterReducer extends Reducer<Text, PageRankElement, Text, Text> {
	@Override
	public void reduce(Text key, Iterable<PageRankElement> values, Context context) throws IOException, InterruptedException{
		String neighbours = null;
		double newPrValue = 0.0;
		double d = Double.parseDouble(context.getConfiguration().get("d"));
		double num = Double.parseDouble(context.getConfiguration().get("num"));
		for (PageRankElement element : values) {
			if (element.isNode()) {
				neighbours = element.getContent();
			} else {
				newPrValue += d * element.getPassValue();
			}
		}
		newPrValue += (1 - d) * (1 / num);
		context.write(key, new Text(neighbours + "PR," + Double.toString(newPrValue)));
	}
}
\end{lstlisting}
\newpage
PageViewer任务是从最后一次迭代的结果中读出人名和$PR$值,然后按照$PR$值从大到小排序输出。在hadoop 框架中,结果会按照Map的key进行排序。所以只要将$PR$值作为key,将人名作为val即可。但是默认的排序是从小到大排序,而我们更关注$PR$值较大的人物。所以重写$PR$值的比较器。除了重写比较器,还需要重写构造器和toString方法。
\begin{lstlisting}[language=Java]
class DecDoubleWritable extends DoubleWritable {
	@Override
	public int compareTo(DoubleWritable o) {
		return -super.compareTo(o);
	}
	public DecDoubleWritable(double val) {
		super(val);
	}
	public DecDoubleWritable() {
		super();
	}
	@Override
	public String toString() {
		return super.toString();
	}
}
\end{lstlisting}

重载比较器后,只要在Map阶段将$PR$值作为键,人名作为值输出即可。在Reduce阶段,将键和值调换顺序输出即可。最终通过一个Driver将三个子任务封装起来,在Driver中也可以根据参数控制迭代次数等。








